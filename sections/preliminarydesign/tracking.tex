\section{Tracking}\label{predesign:tracking}
Tracking can be divided into two parts: determining position and detecting following movement.

\subsection{Determining Position}

Determining a targets position, is a problem which is closely related to detection of a target. In a way, the problem of determining a position is a matter of interpreting the information achieved from the detection, and trying to infer a position based on this. In the case of image recognition it might be as straight forwards as measuring the offset from the detected target to a known point.

When it comes to motion detection, the ability to figure out the position of a detected target can vary a great deal, based on the properties of the sensors. Two variables are to primarily be considered. One is the sensors field of view, and the other is the precision of the information about the position within the sensed area. For example, imagine a sensor with a field of view matching the observable area, which can only give the information \enquote{detected} or \enquote{not-detected}. In this scenario, pinpointing a position is only possible if the velocity of the target is known, which can not be assumed. But lets imagine a very narrow field of view. Suddenly it becomes possible to estimate the position, by moving the field of view through the observable area and taking note of the position when a measurement is taken. The best case scenario however, is a sensor with field of view matching the observable area, which is also able to give precise information about the position of the detected object, in which case the process becomes trivial.

\subsection{Tracking a Target}

Tracking can be modelled as a two step process, prediction and evaluation. The prediction step is the process of interpreting information on a targets position, and trying to predicted its future position. The evaluation step is the process of evaluating a prediction, adding further information for the prediction step. The importance of the two process varies based on different factors. If the information gathered through the sensors are considered accurate and plentiful, then the evaluation step will be a less important source of information to the point of not being of any support. But in cases with little or inaccurate information, the evaluation step can be of great importance, especially if the sensors are not covering the entire observable area, and are directed based on the prediction step. An example of this is the first measurement where a target is detected. At that point in time the prediction step can not know the speed of the target, and the prediction is likely to be wrong. In this case, it will often prove useful to evaluate the first prediction.

Unlike the evaluation step, the prediction step can not be marginalized. The problem with the prediction step is instead a problem of information dependence. It is obvious that a correct prediction will depend on some information, but the amount of information which is required for a correct prediction should be considered the benchmark for any tracking system. That said, the definition of \enquote{correct prediction} can vary, but in this project a correct prediction is considered to be on both position and velocity of the target, since these will be required for meeting all of the requirements. It is also important to note, that although correct predictions are always a priority, sometimes it can prove useful to initially focus on being in proximity of the target, such as with the narrow fielded but movable sensors.

\section{Trajectory Prediction}\label{predesign:shoot}

The last part of the preliminary design will consider trajectory prediction. The actual processes of aiming and shooting will not be covered in the preliminary since they are trivial when hardware factors are not involved, and are even then primarily implementation problems. Trajectory prediction does however deserve a mentioning. Given that the target is being tracked correctly, and thus our predictions on positions and velocity are correct, the right trajectory can be calculated given information on projectile speed and launch delays. But remembering that this process involves mechanical movement which are prone to inaccuracies (at least from a mathematical standpoint) a margin of error is expected. This includes the targets angle relative to the turret, the width of the target, and whether the tracked point of the target is closer to the front of the target or the rear.\\
\eal