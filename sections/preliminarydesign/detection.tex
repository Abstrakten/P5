\section{Detection}\label{sec:prelimdetection}
In \cref{sec:registering} two defining properties of a target were guaranteed. Those properties allow for two ways of detection: image recognition and motion detection.

\subsection{Image Recognition}
Image recognition, as a detection method, involves analyzing a picture of the observed area in order to determine whether a target is present or not. There are two ways to do it: Either the analysis looks for anomalies in the picture, or it looks for something that matches a preconceived signature of a target \cite{PatternRecognition} \cite{Brown:2003:RP:946247.946772}.\\

The first method compares any picture to a baseline, and reacts to changes. This requires a baseline to be established, which can prove to be quite complicated. If a static baseline is assumed, changes in light and shadows can potentially affect the analysis, resulting in false positives. Any instability in the camera mount can also result in the same problem. This could possibly be resolved through filters and margin tolerance. Another method is to have a more fluid system which continuously alters the baseline in accordance with new information. A fluid system would regulate the baseline such that small changes would not register as an anomaly, and only major disruptions from the baseline would. This would warrant a system that intelligently analyzes the input. The idea of a baseline also requires the picture to be taken from a stationary position, or employ an elaborate system to map the 2-dimensional pictures onto a 3rd dimension \cite{Brown:2003:RP:946247.946772}. \\

The second method involves trying to match a predetermined signature in the image. Examples being a specific pattern, such as a square or a circle. This would also allow for registration of multiple types of targets. This method also avoids the problem of the stationary camera and the baseline, but it is still vulnerable to image distortion and false positives. This could happen if squares seen from an angle are not registered or if something which is not a target happens to have a square on it \cite{PatternRecognition}.

\subsection{Motion Detection}
There are multiple ways of employing motion detection, however, due to the properties of the targets, the sensor types will be limited to optical, which was covered as image recognition, ultrasonic and infrared. \\ 

No matter which is used, the methods are similar. One method is to establish a baseline, in the same fashion as image recognition. However, this baseline can be much simpler than the one used for image recognition. One example of a simpler baseline could be: "any object closer than 50 centimeters is considered a target". Once again a baseline could also be fluid, allowing for small changes while still detecting rapid changes. This could be useful in the cases where the baseline is the distance from the sensor to the background, and this distance varies across the observable area. It is also possible to map the entire area, and register anomalies, with the same benefits and difficulties as with image recognition. \\

Motion detection sensors are expected to produce lower amounts of data, which requires less analysis, in a trade-off with a limited possible perception of the actual composition of the area.

\subsubsection{Summary}\label{prelsum}
In summation, detection can be done with both image recognition and motion detection. Image recognition is, however, a more perceptive method and motion detection is considered a lighter method in terms of required computational power.

