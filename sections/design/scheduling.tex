\subsection{Scheduling}\label{sec:scheduling}
nxtOSEK has a built-in scheduler that uses rate monotonic scheduling. There are two different kinds of tasks defined in the OSEK specification: basic tasks and extended tasks. Basic tasks run until they terminate or the operating system switches to another task with a higher priority. An extended task can wait for an event, leaving it in a waiting state. Extended tasks pose a problem for normal rate monotonic models as tasks are no longer independent of each other. For the definations of the variables used in the equations see \cref{tab:FPSdef}\\

\begin{table}[H]
\centering
\begin{tabular}{ll}
\textbf{Variable} & \textbf{Definition}                \\
\hline
$\tau_i$   & i-th task                                    \\
$\tau_i,j$ & j-th instance of $\tau_i$                    \\
$T_i$      & period of $\tau_i$                           \\
$D_i$      & relative deadline of $\tau_i$                \\
$C_i$      & cost (worst-case execution time) of $\tau_i$ \\
$R_i$      & worst-case response time of $\tau_i$         \\
$r_i,j$    & release time of $\tau_i,j$                   \\
$f_i,j$    & response time of $\tau_i,j$                  \\
$d_i,j$    & absolute deadline of $\tau_i,j$              \\     
\hline
\end{tabular}
\caption{FPPS definitions.}
\label{tab:FPSdef}
\end{table}
\FloatBarrier

A utilization based analysis can be used to check if a given set of tasks are schedulable. This is done using the following equation: \\ \begin{center}$U \equiv \sum\limits_{i=1}^N \frac{C_i}{T_i} \leq N(2^\frac{1}{N} - 1)$ \end{center}
\textit{U} is the utilization percentage. \textit{U} converges on the value 0.69 when \textit{N} reaches $\infty$. If the equation does not hold true, it does not mean that the set of tasks are unschedulable. The equation can only be used to determine if a set of tasks are schedulable, it cannot determine if they are unschedulable. In the situation that it is not possible to determine the schedulability of a given set of tasks using the utilization analysis, meaning that $0.69 < U < 1$, a worst-case response time analysis can be used to check the schedulability.

Each task is given a priority based on its period. The task with the smallest period is given the highest priority. In order to find the worst-case response time of a task the equation \\
\begin{center}$R_i = C_i + \sum\limits_{j\in{}hp(i)} \ceil{\frac{R_i}{T_i}} C_j$\end{center}
has to be solved. As $R_i$ is on both the right hand and left hand side of the equation solving it is not possible. Instead, the non-decreasing recurrence relationship \\ 
\begin{center}$R_i^{n+1} = C_i + \sum\limits_{j\in{}hp(i)} \ceil{\frac{R_i^n}{T_j}} C_j$\end{center}
is solved when \\ 
\begin{center}$R_i^n = R_i^{n+1}$\end{center}
is true. The worst-case response time has to be smaller than the period. If this is not the case for a single or more tasks the given set of tasks will not be schedulable because the task(s) will not be allowed to finish before being released again. This analysis is sufficient to determine whether a set of tasks is schedulable - it is also necessary to say that the tasks are schedulable. \\ 

The biggest problem with both of these models is the fact that they do not function if the tasks are not independent of each other. They also do not take content switching into consideration. Both of these problems can be handled by using a model. A model does not require the tasks to be independent and it also supports content switching. The model has to verify that all tasks always reach their deadlines. This is due to the fact that rate monotonic scheduling does not have a choice of what to do next and has to operate with a clearly defined rule. It must not branch out into multiple possible paths. The model-checker has to check that the model never reaches an illegal state in the foreseeable future. The problem with a model is the fact that it is not possible to prove that the created model is actually modeling the given problem.
\addtodo{Someone}{Hvad "designer" vi egentlig her?}

