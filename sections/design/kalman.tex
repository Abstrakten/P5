\section{Kalman Filter}
%%% What problem is the filter expected to solve?
%%% What is the kalman filter?
%%% How does the kalman filter work?
%%% MAYBE: Desciption of the hidden markov model base of the filter.
%%% MAYBE: Derivation of the calculations used in the filter.
%%% How did we derive the matrices used in our system?


%%% Maybe naive Bayesian model
One very simple possible solutions to filtering out sensor inaccuracies would be a simple naive Bayesian network. 

%%% Maybe hidden markov model



%%% What problem is the filter expected to solve?
The sensors used for tracking are not very accurate, and therefore the produces very noisy readings. These noisy measurements may result in a erroneous estimate of the actual position of the target. In order to improve the precision of the estimate, the noisy measurements have to be filtered out. This will be done using a Kalman filter.


%%% What is the kalman filter?
The Kalman filter is based on a hidden Markov model, with the exception that the values in the hidden state are continuous rather than discrete. As the process is assumed to be a Markov process, each hidden state is dependent only on its previous state. As such the values of the future state can be predicted by marginalizing out all the previous state variables from the basic chain rule. In order to use the Kalman filter, first a model of the systems must be formed. This model is given by matrices that are used in the filters calculations.

The Kalman filter consists of a predict and update step. In the predict step, an estimate of the next state is calculated, based on the current state, along with a measure of the uncertainty, assuming a Gaussian distribution, of the predicted state.

In the update step, a reading from the sensors are performed, and this reading is the used to update the prediction of the following state. In order to perform this update, the filter uses a kalman gain. The gain describes how much to trust the measurement, compared to the calculated estimate.