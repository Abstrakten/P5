\section{Real-Time Systems}\label{sec:realtimesystems}\label{\automlabel}

The term real-time systems (RTS) describes systems that must perform certain tasks within a predetermined and limited time frame. There are three major categories in real-time systems: hard, firm, and soft. Real-time systems are classified by the consequence of missing a deadline. In hard real-time systems missing a deadline results in a system failure. In firm real-time systems, missing some deadlines infrequently is acceptable, but a task that missed its deadline is no longer useful. In soft real-time systems missing of deadlines is also acceptable, but the quality of the task that missed the deadline is reduced. \\

The most important attribute of a real-time system is predictability. One needs to be able to know which task is running at any given time. There are several methods that can be used with two of them being examined here, they are cyclic executives and task based scheduling. Cyclic executives are predefined cycles. There are two cycles in cyclic executives, a major cycle and a minor cycle. The major cycle is the least common multiple of the period of all the tasks, while the minor cycle is the greatest common divisor. There are some problems with cyclic executives. The major cycle determines the maximum period of a task, if a task has a cost higher than the minor cycle, it is not schedulable. The big advantage that cyclic executives have, is the fact that it does not require an operating system, only a timing interrupt, and the fact that it is deterministic. \\

Task based scheduling is a method of scheduling an RTS. To ensure that every deadline is met, the tasks have to be analyzed to check their schedulability. There are two methods of scheduling that can be used for a real-time operating systems (RTOS), namely earliest deadline first (EDF) and fixed priority preemptive scheduling (FPPS). EDF is an optimal scheduling algorithm on a primitive uni-processor, meaning that given a set of independent tasks, EDF will give the minimal lateness of tasks. EDF can schedule tasks with a utilisation bound of up too 100\%. Even though it is an optimal algorithm, it is not used often as it is hard to implement. \\

FPPS is another way to schedule a set of independent tasks, with FPPS each task is given a priority, the scheduler will run the task with the highest priority at all time this means that if a tasks is running and a task with a higher priority get released the scheduler with preempt the running task and run the new task. This way of scheduling can schedule tasks with a utilisation bound of around 69\%, if the utilisation bound gets higher low priority tasks might not be done at their deadline. \cite{Scheduling}

\eal