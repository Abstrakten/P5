%%% DON'T INCLUDE - CONTENTS WRITTEN INTO THE SENSOR SECTION
\section{Hardware Architecture}\label{sec:arch}
This section describes the communications between the sensors and the microcontroller. The left side of the blue arrow on \cref{fig:hwArc} shows the internal communication of an infrared distance sensor, and the right side shows the communication from the I$^2$C controller to the main processor. The sensor denoted as $S$ on \cref{fig:hwArc} measures a distance in voltage and uses a conversion table to map the voltage to distance in millimeter. As shown on \cref{fig:hwArc} both values are stored in i2c registers. As described in the last section one of the sensors was faulty, and the distance given by the sensor in millimeters were constant. It turned out to be the conversion table which was broken and caused the wrong distance. The chosen solutions was to implement a new conversion system as described in the previous section.
 The new conversion system are based on a power function taking voltage as input and returning distance in millimeter. The internal conversion table used described measured distance and voltage values to map from voltage to distance. The power function does not use described values and therefor can calculate more precise distances between two described values of the conversion table.
A power functions turned out to be more accurate than the conversion table. Further more the greatest benefit of using a power function was the voltage register of the i2c controller was updated more often than the distance register, which makes it possible increase the sample rate of the distance measurements. 

\begin{figure}[H]
\begin{center}
\begin{tikzpicture}

    \draw (0,8) -- (1,8) -- (1,7) -- (0,7) -- (0,8);
    
    \node at (0.5,7.5) {S};
    
    \draw [-triangle 90,fill=black](0.5,7) --  (0.5,2);
    
    \draw (0.7,7) -- (0.7,4);
    
    \draw [-triangle 90,fill=black](0.7,4) -- (1.5,4);

    \draw (0,2) -- (2,2) -- (2,1) -- (0,1) -- (0,2);
    \node at (1,1.5) {i2c slave};
    
    %table
    \draw (1.5,5) -- (3.4,5) -- (3.4,3) -- (1.5,3) -- (1.5,5);
    \draw (1.5,4.5) -- (3.4,4.5);
    \draw (1.5,4.15) -- (3.4,4.15);
    \draw (1.5,3.8) -- (3.4,3.8);
    \draw [dotted, line width=1pt](2.50,3.6) -- (2.50,3.2);
    \node at (2.4,4.7) {Converter};
    
    \draw [-triangle 90,fill=black](1.7,3) -- (1.7,2);
    
    \draw [triangle 90-triangle 90,fill=green, color=blue](2,1.5) -- (4,1.5);
    
    \draw (4,2) -- (6,2) -- (6,1) -- (4,1) -- (4,2);
    \node at (5,1.5) {i2c master};
    
    \draw [triangle 90 -triangle 90,fill=black](5,2) -- (5,3);
    
    \draw (4,4) -- (6,4) -- (6,3) -- (4,3) -- (4,4);
    \node at (5,3.5) {arm/cpu};

\end{tikzpicture}
\end{center}
\caption{find caption}
\label{fig:hwArc}
\end{figure}

