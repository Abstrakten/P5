\section{Actuators}\label{sec:actuators}\label{\automlabel}
The system has to be able to track a target which requires the use of a motor that will allow the turret to rotate. Aalborg University provides the \textit{Interactive Servo Motor} from LEGO Mindstorms as the only LEGO NXT compatible motor. The motor has an encoder which tracks how far the motor has moved. The encoder supports two levels of precision: 360 ticks, which offers precision down to a single degree, and 720 ticks which offers precision down to half a degree. The standard library uses the 360 tick level of precision \cite{NXTMOTOR}. The standard library  has a function which instructs the motor to turn a given number of degrees. This function was tested and found to overshoot the target by approximately 70 degrees. As this level of inaccuracy is quite high, it becomes necessary to find a solution that allows the motor to turn more precisely. \\

The performance of the motors was tested by connecting them to an NXT and executing a program which counts the revolutions of the motor at variating power levels - from 10\% to 100\% in increments of 10\%. Data was collected for 10 seconds at every interval. The motors were placed in an upright position and without any external load affecting them. Two different battery solutions were available for the NXT: a 7.4V rechargeable battery or six 1.5V batteries with a total output of 9V. The six battery solution was selected as the power of the motors was increased with an output of 9V compared to 7.4V.

\begin{figure}[H]
\centering
\begin{tikzpicture}
    \begin{axis}[title=Motor Performance Test, xlabel=Power level \%, ylabel=RPM, ymajorgrids=true, grid style=dashed, width=14cm, height=9cm, domain=0, legend pos=north west, legend cell align=left, xmin=10, ymin=0]
    \addplot[yellow,line width=1pt] 
         table [x=a, y=b, col sep=comma] {sections/data/MotorData.csv};
         \addlegendentry{Motor A}
    \addplot[red,line width=1pt] 
         table [x=a, y=c, col sep=comma] {sections/data/MotorData.csv};
         \addlegendentry{Motor B}
    \addplot[blue,line width=1pt] 
         table [x=a, y=d, col sep=comma] {sections/data/MotorData.csv};
         \addlegendentry{Motor C}
    \addplot[purple,line width=1pt] 
         table [x=a, y=e, col sep=comma] {sections/data/MotorData.csv};
         \addlegendentry{Motor D}
    \addplot[orange,line width=1pt] 
         table [x=a, y=f, col sep=comma] {sections/data/MotorData.csv};
         \addlegendentry{Motor E}
    \addplot[green,line width=1pt] 
         table [x=a, y=g, col sep=comma] {sections/data/MotorData.csv};
         \addlegendentry{Motor F}
    \addplot[black,line width=1pt] 
         table [x=a, y=h, col sep=comma] {sections/data/MotorData.csv};
         \addlegendentry{Motor G}
    \addplot[teal,line width=1pt] 
         table [x=a, y=i, col sep=comma] {sections/data/MotorData.csv};
         \addlegendentry{Motor H}
    \addplot[lime,line width=1pt] 
         table [x=a, y=j, col sep=comma] {sections/data/MotorData.csv};
         \addlegendentry{Motor I}
\end{axis}
\end{tikzpicture}
\caption{Motor test}\label{motordata}
\end{figure}
\FloatBarrier

\Cref{motordata} shows the collected data from the motor tests. The x-axis denotes the power level percentage and the y-axis denotes the revolutions per minute (RPM) of the motor. The tests showed that motor \emph{E} and \emph{C} behaved abnormally compared to the other motors. At a certain point the motors started slowing down as more power was supplied rendering the two motors unusable. The cause of this issue is suspected to be an internal motor issue. The other motors have different power levels at which they start rotating. The motors \emph{D} and \emph{G} have the lowest rotation starting points at 30\% power and 40\% power respectively. For all motors except \emph{E} and \emph{C}, the RPM rises steadily in a similar pattern as more power is supplied. 

