\section{Actuators}\label{sec:actuators}
The system has to be able to aim at a target which requires the use of a motor that will allow the tower to rotate. Aalborg University provides the \textit{Interactive Servo Motor} from LEGO Mindstorms. The motor has an encoder which tracks how far the motor has moved. The encoder supports two levels of precision: 360 ticks, which offers precision down to a single degree, and 720 ticks which offers precision down to half a degree. When the motor is instructed to turn a given number of degrees the precision is limited. The motor starts turning towards the degree it was given, however, when it reaches the degree it does not stop, as there exists no brake. This functionality was tested and found to overshoot the target by as much as 70 degrees. As this level of inaccuracy is quite high, it is necessary to find a solution that allows the motor to turn more precisely. The performance of the motors was tested by measuring the revolutions per minute (RPM) of the motor at variating power levels - from 10\% to 100\% in increments of 10\%, each interval lasting 10 seconds. The motors were placed without any external load affecting them. Two different battery solutions were available for the NXT: a 7.4V rechargeable battery or six 1.5V batteries with a total output of 9V. The six battery solution was selected as the power of the motors was increased by 1.6V going from 7.4V to 9V.

\begin{figure}[H]
\centering
\begin{tikzpicture}
    \begin{axis}[title=Motor Performance Test, xlabel=Power level \%, ylabel=RPM, ymajorgrids=true, grid style=dashed, width=14cm, height=9cm, domain=0, legend pos=north west, legend cell align=left, xmin=10, ymin=0]
    \addplot[yellow,line width=1pt] 
         table [x=a, y=b, col sep=comma] {sections/data/MotorData.csv};
         \addlegendentry{Motor A}
    \addplot[red,line width=1pt] 
         table [x=a, y=c, col sep=comma] {sections/data/MotorData.csv};
         \addlegendentry{Motor B}
    \addplot[blue,line width=1pt] 
         table [x=a, y=d, col sep=comma] {sections/data/MotorData.csv};
         \addlegendentry{Motor C}
    \addplot[purple,line width=1pt] 
         table [x=a, y=e, col sep=comma] {sections/data/MotorData.csv};
         \addlegendentry{Motor D}
    \addplot[orange,line width=1pt] 
         table [x=a, y=f, col sep=comma] {sections/data/MotorData.csv};
         \addlegendentry{Motor E}
    \addplot[green,line width=1pt] 
         table [x=a, y=g, col sep=comma] {sections/data/MotorData.csv};
         \addlegendentry{Motor F}
    \addplot[black,line width=1pt] 
         table [x=a, y=h, col sep=comma] {sections/data/MotorData.csv};
         \addlegendentry{Motor G}
    \addplot[teal,line width=1pt] 
         table [x=a, y=i, col sep=comma] {sections/data/MotorData.csv};
         \addlegendentry{Motor H}
    \addplot[lime,line width=1pt] 
         table [x=a, y=j, col sep=comma] {sections/data/MotorData.csv};
         \addlegendentry{Motor I}
\end{axis}
\end{tikzpicture}
\caption{Motor test}\label{motordata}
\end{figure}
\FloatBarrier

\Cref{motordata} shows the collected data from the motor tests. The x-axis denotes the power level percentage and the y-axis denotes the RPM of the motor. The other motors have different power levels at which they start rotating. The motors \emph{D} and \emph{G} have the lowest rotation starting points at 30\% and 40\% power, respectively. For all motors except \emph{E} and \emph{C}, the RPM rises steadily in a similar pattern as more power is supplied. Motor \emph{E} and \emph{C} behaves abnormally compared to the other motors. At a certain point the RPM of these motors starts dropping as more power was supplied, rendering the two motors unusable.

\subsection{Motor Selection}
Based on the motor test the most suitable motors were selected for the tower. According to the motor test the motors \textit{D} and \textit{G} are the best with the remaining motors being nearly equal in performance - with the exception of \textit{E} and \textit{C}. Motor \textit{D} has been selected as the rotation motor due to its performance, the early rotation starting point being important. Motors \textit{G} and \textit{B} have been selected for the cannon. The cannon only requires motors with similar performance from above 60\% power.\\


