\chapter{Discussion}
This chapter will discuss some of the choices made throughout the project, provide alternative solutions to problems that were encountered, and contemplate the changes a different focus could have had.
% Better?
%This chapter will contemplate different ideas that either would have been implemented given more time, a different choice of focus, or simply possible alternative solutions to problems in the project. 

%% Kan vi tage dem kronologisk?
\section{Target Focus}
In the analysis it was decided to focus solely on the turret, and thus the targets were chosen as being simple passive objects, moving along the track, not capable of communicating with the turret in any way. This choice resulted in a limitation of the detection methods available, along with a limitation of the turret not being able to register whether a target is hit or not. If the targets had been more in focus, the delimitation might have allowed for other ways of tracking the targets, such as radio-frequency identification, Bluetooth or simply other ways of utilizing the currently used technologies. Instead of having the turret read a reflection of an infrared signal, it could have been detecting a signal sent from the target. This could have provided more certainty to the tracking system. Then perhaps the tracking system could have worked with a different interpretation of the Markov process, and as such dealt with more complex movement patterns than a constant speed.\\

%Alternatively, if the targets were considered too predictable, providing them with a level of intelligence could be considered. This could manifest in different abilities, such as changing speeds while running, changing behavior when hit, or maybe letting them navigate through multiple possible paths. This could create the illusion that the towers have an indirect effect on the targets, such as slowing or completely halting them for a few seconds, but it could also provide a greater challenge for the turrets, which in turn can give the turrets an opportunity to show off their abilities.\\

%Having targets that actively communicate with the turret could be considered as an alternative. By letting the targets constantly broadcast their positions to the turret, a more refined detection method could have been implemented. It would be possible for the turret to detect targets without them having to pass in front of the turret, along with it being able to identify the position of multiple targets simultaneously. It could also result in targets being detected before entering the zone in which the turret is able to aim, allowing for better tracking as the turret could analyze the targets movement before entering this zone. This would, however, pose a challenge as the turret must be able to determine its own position relative to the positions it receives from the targets. It would also be possible to equip the targets with means of registering whether they have been hit by a projectile from the turret, and this information could then be communicated to the turret. This way the turret would be able to ignore targets that have already been hit, and evaluate its performance in hitting targets. It would also make it possible to train the turret's speed, detection and tracking model.  \\


%The targets of the project were chosen from a perspective of being enemies of the turret, and as such should not be expected to help the turret in any way. This decision also had a secondary effect of making the project require a non-trivial tracking property. While a game of Tower Defense gives the player the impression that the towers are on his or her side, ultimately, this is only an illusion, as every object of the game is exactly that, an object, and thus does not have any will or stake in the game. Therefore, it would not have been against the spirit to have the targets information with the turrets \addtodo{Mathias}{Huh?}, so that the challenge is not between the targets and the turrets, but remains between the player and the pieces. This could have included some sort of signaling, either with complete or partial information on the state of the target. \\

%But even if it is deemed wishful to keep information hidden from turrets, targets might at least have a way to register and/or broadcast if they have been hit. This stops the turrets from having to physically stop the targets, which requires a very different approach to the challenge. \\


\section{Choice of LEGO NXT}
The choice of using the LEGO NXT as the platform was made primarily to facilitate the construction of the turret using LEGO and the ease of connecting sensors. While these aspects were indeed made easier, this choice also resulted in limitations for the turret in the form of limited selection of available sensors and actuators. These limitations resulted in the selection of sub-optimal sensors and actuators for the turret, negatively affecting its accuracy. Consequently, a lot of time was spent on correcting and accounting for these inaccuracies. Using cameras as sensors was also excluded due to choice of the LEGO NXT platform as it would require offloading the necessary computations. \\

While another platform may have required modifications of sensors, this could have been justified due to the availability of better components. Additionally, had a more powerful platform been chosen, using cameras and image recognition could have been an option. The option of using cameras and image recognition for detection and tracking could have improved the functionality of the turret. It would have made it possible to distinguish between different targets, and also distinguish non-targets from targets. Image recognition would also provide more accurate information about where in the turret's field of view targets were located, and allow for the turret to detect multiple objects within its field of view. 

\section{Sensor Choices}
The sensors chosen in this project were chosen because the alternatives were discarded. It is, however, worth considering alternative ways of utilizing the sensors. The main issue with the ultrasonic sensors was the interference with other ultrasonic senors, but it would not have had the same problem if a single ultrasonic sensor had been used alongside infrared sensors. The possibility of mounting two infrared sensors on top of each other, with an ultrasonic sensor as well, such that all three sensors would have had their field of view centered on the same point, could potentially have had a positive impact on the accuracy. The wide cone of the ultrasonic sensor could have provided the broad view, while a cross-examination of the more narrow infrared sensors could potentially have increased the accuracy. \\

In \cref{sec:dessensor} it was discovered that it is possible to determine different detection zones based on the variances in the distance readings. We discovered that, to some degree, the distance readings drop in a linear pattern when the target moves closer to center of the sensor cone. In case this hypothesis holds it would be possible to construct a function calculating the offset relative to the center of the cone. This could result in a more accurate calculation of the target's position due to less discretization of the sensor area described in \cref{sec:dessensor}. Due to inconsistencies in the sensor the hypothesis may not hold, therefore, this functionality was considered to have a lower priority, and thus was not implemented. 

\section{Meta Optimization}
Throughout the project there has been a need for optimization of both the PID controller, \cref{design:PID}, and the Kalman filter \cref{sec:design_tracking}. The tuning of these has been done by hand, by changing values and estimating the effects of these changes. This is not to say that it was done without a method or that the results are not adequate, but simply that the process was demanding and that the results possibly could have benefited from an automated process. For the PID controller, making an intelligent optimization program could have been done quite simply, since the data was available and specifying a goal would be a mix between the lowest steady-state-error, settling time and rise time. \\

The Kalman filter might have benefited from meta-optimization by finding the optimal initial values for the different matrices. However, this would have been more difficult for a number of reasons. Firstly, defining a goal condition would have been hard, since, at the current state, it is not possible to register a hit in the system. Instead a goal condition could have been defined as "maximum time at zone 0 during tracking". This might have worked, but could possibly result in false positives since it would be based on the sensors' readings, and not the actual position. In addition to this, a possible over-fitting could happen, since the system is not aware of the levels of speed, but interprets them as continuous. The benefits of a good meta-optimization would in return be quite good, since it would allow for more precise calibration, which, considering the commercial aspect, would provide a massive bonus, as the turret would not need to be manually set up, but rather just put in calibration mode, or constantly revise its own calibration.

\section{Scheduling}
The scheduling algorithm was chosen as it was an optimal scheduling algorithm for fixed priority scheduling, and better alternatives would be difficult to implement, as they require additional information about the worst-case running time. The utilization test are working under some assumptions, one of these is that it takes no time to switch from one task to the next, this is not true on any processor. The assumptions also limit design choices, as tasks have to be independent, which means that they cannot be reliant on each other. A system model is not necessary limited by these assumptions and guarantees could still be made regarding the schedulability of the system. If another scheduling algorithm had been chosen, it might have been possible to make the remaining 22\% of CPU available to the system which would make it possible to gather more sensor readings, increasing the accuracy of the estimate of the target's position.



%\section{Game Mechanics}
%Following the discussion of the targets, a wide array of other game aspects that were not included in the project, but could easily have changed it, deserve at least a short mentioning. First, a working tower should not be confined to just reacting to targets coming from one direction. While targets can always be expected to move from one end of the track to another, it would be quite limiting to only have a tower working while standing on the correct side of the track. Second, the current setup could easily be considered too dependent of its distance to the track. If a tower is confined to a very small variation in the range, it limits the possible combinations or positioning, which is one of the key attractions of a Tower Defense game. This also extends to other aspects of placement, such as placement in or around corners of the track, as well as places where a tower should be able to see the target passing by multiple times, such as U-turns, or hairpin curves. Features which allow these obstacles would possibly require some user input or at least some calibration in order to work properly, however, they still represent important parts of what makes room for creative strategic thinking in the game.






%Meta optimization
%sensor choice
%Tracking (recognition targets)
%scheduling
%shooting left/right
%offset
%automatisk afstand
%turret kan se flere stykker af trackket
%intelligente targets
%flere dimensioner i kalman/extended kalman/vores kalman er en lilleput kalman