\section{PID}\label{imp:PID}
The main PID function can be seen in \cref{PID}. The function takes four variables as input: the first two being the target position and the current position of the turret, and the final two being the integral and the last error. The latter two are both variables used to store data from the last run of the PID controller. They are not stored in the function itself because that would cause a problem with the PID controller running multiple motors. \\

The \emph{PID} function utilizes four defines that are located in the PID.h file. These defines are $K_p$, $K_i$, $K_d$ and \emph{MotorStartPower}, as seen in \cref{PID.h}. The first three are coefficients for the PID controller, as explained in \cref{design:PID}, and \emph{MotorStartPower} is the minimal power required for the motor to rotate. The variables $K_p$, $K_i$ and $K_d$, were found using the Ziegler–Nichols method and then tuned manually. $K_i$ is much smaller than the other two because the \emph{I} value in the PID controller is the sum of all the errors that the system has seen, and with a sample rate of 20 milliseconds, \emph{I} will sum up fast. \emph{MotorStartPower} is set to 80, which is the required power level when the motor is running under load.

\begin{lstlisting}[style=customc, label={PID.h}, caption={PID.h}]
#define Kp 0.075            // Kp
#define Kd 0.0375           // Kd
#define Ki 0.0000000075     // Ki
#define MotorStartPower 80

extern S32 PID(S32, S32, S32*, S32*);
extern S32 MotorPID(S32 target, U8 motor, U8 flag);
\end{lstlisting}


\begin{lstlisting}[style=customc, label={PID}, caption={PID code}]
S32 PID(S32 target, S32 current, S32 *integral, S32 *lastError)
{
	S32 error = target - current;
	S32 derivative = error - *lastError;
	*integral = *integral + error;
	double speed = error * Kp + Ki*(*integral) + Kd*derivative;
	*lastError = error;

	return (speed > 0) ?
        (S32)(speed + MotorStartPower) :
        (speed < 0) ?
        (S32)(speed - MotorStartPower) :
        (S32)speed;
}
\end{lstlisting}

\Cref{PIDwrapper} shows the wrapper for the \emph{PID} function. The function has two arrays with a size of three. These are used to hold the \emph{integral} and \emph{lastError} variables. The wrapper takes two variables as input: the first is the motor's target position, and the second is the number of the port the motor is connected to. Because the NXT has three output ports the length of the array is three. The port numbers are represented by integers in the range zero to two. The wrapper then retrieves the current position of the motor, and calls the \emph{PID} function. Finally, it sets the speed of the motor to the value returned by the \emph{PID} function, and returns this value, shown in \cref{PID}. The PID wrapper has two modes that are determined by the value of the variable \emph{flag}. If the flag is not set, the PID will be in weapon mode and it is allowed to miss the target by four degrees. If the flag is set the PID will be running all the time and no error margin is allowed. This mode is used for the targeting system as it needs a constant stream of information from the PID. The array \emph{lastTarget} holds the last target of the PID, and if the target changes the PID will reset.

%if flag is false it is in weapon mode and is allowed to miss the target by +/- 4 degrees. If flag is true it is running all the time and not allowed for an error margin, this is used for the targeting system as it need to get constant calculation and are not satisfied with a small error margin. lastTarget hold the last target so that if the target change the PID will reset.

\begin{lstlisting}[style=customc, label={PIDwrapper}, caption={PID wrapper}]
S32 MotorPID(S32 target, U8 motor, U8 flag)
{
    static S32 getSpeed = 0;
    static S32 integrale[] = {0, 0, 0};
    static S32 lastError[] = {0, 0, 0};
    static S32 lastTarget[] = {0, 0, 0};

    if(lastTarget[motor] != target)
    {
        integrale[motor] = 0;
        lastError[motor] = 0;
        lastTarget[motor] = target;
    }

    S32 current = nxt_motor_get_count(motor);
    current = (flag)? -current : current;
    if( flag || !((getSpeed <= MotorStartPower && getSpeed >= -MotorStartPower) && (current - target) <= 4 && (current - target) >= -4)){
        getSpeed = PID(target, current, &integrale[motor], &lastError[motor]);
        getSpeed = (flag)? -getSpeed : getSpeed;
        nxt_motor_set_speed(motor, getSpeed, (!getSpeed))
    }
    else 
    {
        nxt_motor_set_speed(motor, 0, 1);
    }
    return getSpeed;
}
\end{lstlisting}


