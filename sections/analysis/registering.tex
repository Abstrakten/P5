\section{Detection}\label{sec:registering}

The first part of creating a tower is to detect that an object is present in the observable area. The observable area is the entire area which a given tower can possibly detect a target. This can be done with a wide array of sensors and techniques. However rarely, just registering that an object is present somewhere is not enough, and as such, basic information is required. Firstly there is the matter of distinguishing between a target and a non-target. Considering the delimitation of the Lego train, an example of a non-target could be the tracks which the train runs on. We must therefore define some properties of a target, from which we can identify it.

A target will always be:

\begin{enumerate}
\item Moving along a track inside the observable area.
\item Visually distinguishable from non-targets. 
\end{enumerate}

Detecting a target in the observable area will require monitoring of said area. The direct approach is to constantly cover the entire observable area, but a tower might not have such a large field of view. Field of view is defined as the area which a tower can observe at a given moment. If complete constant coverage is not a viable solution, sensors can either watch key points in the area, or be moving the field of view around the observable area in some manner.