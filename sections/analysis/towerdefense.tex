\section{Tower Defense}\label{sec:towerdef}
% Hvad er tower defense? Hvilken type tower defense laver vi et tårn til? Hvad er kravene her?
Before the analysis of the different problems of detection, tracking and shooting, an explanation of the Tower Defense genre will be given, and the type of game the solution will primarily be inspired by, will be identified.\\

The tower defense genre is popular, with countless different versions being produced. The most played game on the site Armorgames.com alone has been played 64 million times \cite{td1}. The purpose of the game is to prevent enemies from reaching a certain point on a map. This is done by placing defensive towers, usually different types, on the map, that will attack approaching enemies. \Cref{towerdefimg} shows a screenshot of a tower defense game. The amount of towers available to the player is limited, making it a challenge to strategically place the towers around the map, so that their different strengths are utilized.

\imgscale{figures/towerdefenseedited.png}{Simple tower defense game \cite{td2}}{towerdefimg}{0.42}

Tower defense games often adopt different tower placement policies. Some games have maps with a predefined path that all enemies will follow. The player must then place their towers alongside this path. This is the policy used in the game seen in \cref{towerdefimg} where the path is represented by the red bricks, everything placed on the green area is a tower, the enemies are the tanks on the path, and the destination is seen as a large building in the top right of \cref{towerdefimg}. Other games do not have a set path, but instead rely on the player creating a maze using towers. The enemies must then navigate through the maze before they can make it to their destination. This project will adopt the policy of a predefined path along which the enemies must travel. \\ %This means that the enemies will always be coming from the same direction, effectively allowing all towers to wait in the most optimal position for intercepting incoming enemies. \\ 

\label{targetdelim}The target chosen for this project is a LEGO train. This was selected because it can travel along a predefined path of tracks. The speed of the train can also be adjusted, and because it is made of LEGO bricks the train can be modified to fit the needs of the project. These properties were deemed sufficient for a choice of target, since the navigation of the target is not a primary concern of this project. The train is able to move with varying speed, but it will be limited to a constant speed while passing the tower. Since the target will pass multiple times, the predefined path will be circular, and the target is delimited to always travel counterclockwise around the track.



% Tower defense intro: before detection, tracking, etc. we will explain td...
% Move to introduction -> 1.1


%Motivation for this project is drawn from the popular sub-genre of video games called tower defense. In a game of tower defense several enemies move along a predetermined track, and different towers are then placed along the track, such that they can shoot at the enemies. The goal of the game is to set up the different towers available to you, in such a way that no enemy will reach the end of the track. This is made challenging by giving towers different properties. Examples could be one tower which aims badly but inflicts lot of damage if it hits, while another tower has a long range, fast rate of fire, but a small observable area. There are countless variations to the standard game, but we choose to focus on these basic parts of the game.\\