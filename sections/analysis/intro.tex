\chapter{Analysis}\label{ch:analysis}\label{\automlabel}
%Describe the purpose of the analysis chapter
In order to create a real world implementation of tower defense, we must examine the underlying problem. There are of course many ways to create such a game, but we choose to view the targets as unintelligent, and as such trivial to implement.\\

Then there is the matter of implementing towers. In computer games, the towers act autonomously, and they can instantly know of any targets within its range. It is also trivial to aim and hit targets, which in the games are mostly done for visual appeal, and not actually a requirement. \\

In a real world implementation, the towers should remain autonomous, and as such, the points of surveying an area for targets, aiming, and hitting your targets, are much more complicated. \\

The processes of a tower can therefore be split into three parts: detection, tracking, shooting, which in turn can be divided further into sub-parts. This chapter examines the challenges that compose this problem. \\

The target chosen for this project is a LEGO train. This was selected because it can travel along a predefined path of tracks. The speed of the train can also be adjusted and because it is made of LEGO the train can be modified to fit the needs of the project. These properties were deemed sufficient for a choice of target, since the movement of the target is not a primary concern of this project. The train is able to move with varying speed, but it will be limited to a constant speed while passing the tower. Since the project will not concern itself with the task of optimising the enjoyment of the game, we estimate that a realistic speed for a target is between 0.2 m/s and 1 m/s.\addtodo{Kasper}{er denne begrundelse for valgt hastighed tilfredsstillende}

\section{Tower Defense}\label{sec:towerdef}
% Hvad er tower defense? Hvilken type tower defense laver vi et tårn til? Hvad er kravene her?
Tower defense is sub-genre of the real-time strategy genre \cite{td1}. The purpose of the game is to prevent enemies from reaching a certain point on a map. This is done by placing towers, usually different types, on the map. These towers will then shoot at the enemies, killing them and granting the player currency which can be used to purchase more towers or upgrade towers already placed on the map. See \cref{towerdefimg} for an example of a tower defense game. The enemies, available towers and map varies from one game of tower defense to another. The strategy element of the game lies in the placement of the towers around the map. The player must attempt to place the correct types of towers in the most suited locations in order to stop the enemy from advancing. 

\imgscale{figures/towerdef.png}{Tower Defense: Lost Earth \cite{td2}}{towerdefimg}{0.3}

Tower defense games often adopt different tower placement policies. Some games have maps with a predefined path that all enemies will follow. The player must then place their towers along this path. This is the policy used in the game seen in \cref{towerdefimg}. Other games do not have a set path, but instead rely on the player creating a maze using towers, the enemies must then navigate through the maze before they can make it to their destination. \\

This project will adopt the policy of a predefined path along which the enemies must travel. This means that the enemies will always be coming from the same direction, effectively allowing all towers to wait in the most optimal position for intercepting incoming enemies.

% Target selection from analysis introduction moved here
% Tower defense intro: before detection, tracking, etc. we will explain td...
% Move to introduction -> 1.1
\section{Detection}\label{sec:registering}

The first part of creating a tower is to detect that an object is present in the observable area. The observable area is the entire area which a given tower can possibly detect a target. This can be done with a wide array of sensors and techniques. However rarely, just registering that an object is present somewhere is not enough, and as such, basic information is required. Firstly there is the matter of distinguishing between a target and a non-target. Considering the delimitation of the Lego train, an example of a non-target could be the tracks which the train runs on. We must therefore define some properties of a target, from which we can identify it.

A target will always be:

\begin{enumerate}
\item Moving along a track inside the observable area.
\item Visually distinguishable from non-targets. 
\end{enumerate}

Detecting a target in the observable area will require monitoring of said area. The direct approach is to constantly cover the entire observable area, but a tower might not have such a large field of view. Field of view is defined as the area which a tower can observe at a given moment. If complete constant coverage is not a viable solution, sensors can either watch key points in the area, or be moving the field of view around the observable area in some manner.
\section{Tracking}\label{sec:tracking}\label{\automlabel}
Tracking a target, is closely related to detection of a target. However they differ in the amount and type of information they require. Tracking a moving target can be a very complicated procedure as there are many factors to consider. These factors can be the speed of the target, the direction of the target, the speed of the tracking device, etc.\\

The targets movement is obviously the greatest challenge. The target may be moving at a constant speed, but its speed can also vary. The tracking system needs to react to these changes in speed quickly to avoid losing sight of the target, especially if the target is moving at high speed. Part of this problem is removed in this project as the target will always remain at a constant speed while passing the tracking system. However, this speed may change in-between passes. The direction of the target is also a part of the challenge. The tracking system will have to determine the direction in which the target is moving to ensure it isn't moving the offensive systems of the tower away from the target. The distance to the target also plays an important role. The closer to the target the tracking system is, the shorter the time span in which the target is within its observable area is.\\

The act of tracking ultimately boils down to detecting movement, analysing movement, and reacting accordingly.\eal
\section{Shooting a Target}\label{sec:shooting}
For this project, the shooting part will be the launch of a projectile, which gives certain challenges when attempting to shoot a moving target. \\

Certain factors have to be considered when firing projectiles at any given target. The most important factors are the velocity of both the target, and of the projectile, as well as the distance to the target and the accuracy of the projectiles. Another important factor is the predictability of the target, mostly in regards to the path which the target will take. However, due to the delimitations to the tower defense scenario in this project, a predictable path is assumed. \\

The properties of the tower that was eventually constructed will be covered in \cref{turret}, but some deductions can be made from the other factors alone.

\subsection{Field of View}\label{FoW}
As the target enters the field of view, the shooting system has a limited time to respond, where the time available depends on the speed of the target. A faster target will leave the field of view quicker, providing a shorter time frame for the projectile to hit the target. If a degree of freedom is assumed in the distance between a tower and a target, the time that target spends in the field of view can be increased at the cost of increasing the travel time of the projectile.\\

The distance from a tower to a given target inside its observable area, is not likely to be constant. In fact this is only the case when the target follows a path which curves in parallel to the tower. Assuming a straight path parallel to a tangent of a towers field of view, the distance to a target will increase in relation to the distance between that target and the tangent's point of contact. This fact, in relation to the velocity of the projectile, limits the maximum distance to the target, and as such also the effective observable area. This can be seen in \cref{25} where $a$ shows the shortest distance to the track, $b$ shows the maximum shooting distance and $c$ shows a distance out of reach. This shows that as $a$ shortens, the effective range increases, but never more than the range of $b$. On the other hand, increasing the field of view by extending $a$, happens at the cost of a diminished observable area.

\begin{figure}[H]
\begin{center}
\begin{tikzpicture}

   \draw [dashed] (3,0) arc (0:180:4cm);
   \draw [thick](-5,3) -- (3,3);
   \draw [-triangle 90,fill=black](-1,0) -- node[left] {a} (-1,3);
   \draw [-triangle 90,fill=black](-1,0) -- node[left] {b} (0.8,3.6);
   \draw [-triangle 90,fill=black](-1,0) -- node[left] {c} (3,3);
   %\draw (-0.9,2.8) -- (-1.1,2.8) --(-1,3) -- (-0.9,2.8);

\end{tikzpicture}
\end{center}
\caption{Illustration of the effective observable area}
\label{25}
\end{figure}
\smallskip

The limit on the time a target spends in the effective range of a tower, also adds a requirement of time to the task of acquiring a target and releasing a projectile. If targets require multiple hits before being eliminated, which is the case in classical tower defense games, this will further increase the emphasis on time, since the faster the projectiles are fired, the sooner that target will be eliminated. However, for the purpose of a game, this could be considered a part of the challenge of managing your tower resources. So a tower should not necessarily have the fastest possible reaction, but rather a consistent one. 

\subsection{Trajectory Prediction}\label{sec:motionprediction}
When firing projectiles the correlation between the ratio of the projectiles travel time, and the time it takes for the target to travel its own length, must be considered. If for example the distance is one meter, and the projectile velocity is one meter per second, a target that moves faster than its own length in a second, will not be hit by a projectile launched directly at the target. If we assume a passing target to move at a pace sufficiently high, such that a tower can not successfully hit the target by firing immediately upon detection, it becomes necessary to be able to predict the movement of the target, and fire the projectile a distance in front of target.\\

In summation, the process of firing a projectile is a time sensitive task, in regards to both how long the target is in the effective range of the tower, how fast a projectile can be released, and how fast a prediction of the trajectory can be made.
\bigskip





%A possible solution for this is measuring the speed and direction of the target, and the distance from the turret to the target, at one fixed point. Based on these measurements the turret should calculate how far the target will travel in a given time frame, and then adjust its aim according to this calculation. As this method bases the calculations on a single measurement, it has certain limitations. It is not able to account for change in the targets speed or direction after the measuring point. In order to circumvent these limitations, another method for measuring characteristics of the target is necessary.

%As the limitations stem from basing the calculations on a single measuring point, other measuring points could be added. As the turret does not know the path of the track beforehand, there is a possibility of the track turning out of the turrets range between measuring points. In order to avoid this, the measuring points can not be too far from each other. A large number of measuring points will increase the precision of the prediction of the targets movement. One possible method to increase the amount of measuring points, is to let the turret continuously track its target, while also measuring the distance.


%This distance depends on several different factors, including the objects speed, the distance from the cannon to the object, and the direction in which the object is moving. In order to calculate this distance, SOMESTUFF will be used.
%\addtodo{Lasse}{What stuff?}
%\addtodo{Lasse}{The angle the cannon has to turn depends on both the calculated distance, and the path of the track.} 
%\eal

%\subsection{SOMESTUFF}



%Below are some examples, that illustrate some of the problems that need to be considered. \\

%\textbf{Example 1} \\
%The turret is placed at a specific distance from the path of the moving target. The faster the target moves the faster it will go past the tracking system. If the turret is moved close to the path of the target the field of view of the tracking system will decrease. This results in a shorter time span in which it's possible to hit the target with a projectile.  \\\\

%\textbf{Example 2} \\
%The target is moving at the same speed at all times and its size does not change. The turret is moved in a straight line towards and away from the path of the target. The farther away the turret is moved, the more important the speed and drop of the projectile becomes. At long ranges the projectile might become inaccurate, partly because the travel time increases. \\\\

%\textbf{Example 3} \\
%The turret is placed at a specific distance from the path of the target. The speed of the target can change at any time. If the turret is placed far away from the path of the target the projectile will have a long travel time. If the speed of the target changes at the moment the projectile is fired, the prediction of the target's location may no longer be accurate which can result in the projectile missing the target.